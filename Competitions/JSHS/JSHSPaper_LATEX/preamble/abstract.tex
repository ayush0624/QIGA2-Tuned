\vspace*{5cm}
\addcontentsline{toc}{section}{Abstract}
\renewcommand{\abstractname}{\color{MyDarkBlue}{Abstract}} 
\begin{abstract}
Although computers have increased in computational ability, one category of problems, the NP-Complete problem set, still alludes modern technology. However, quantum computing, a computational method that utilizes quantum mechanics to overcome the limitation of a two-state system, has emerged as a potential solution. The uniqueness of this researcher's current work, Quantum Save, starts by optimizing the Knapsack Problem. Using Qiskit Toolkit and IBM Q32 Simulator, an evolving qubit population was modified using a quantum genetic algorithm. Quantum Save implements a unique disaster algorithm along with tuned hyper parameters for probability amplitude manipulation (a reproduction characteristic). Using this algorithm, knapsack problem optimization increased by 3.75\%. Due to Quantum Save’s promising results, it was applied to budget allocation for smaller-medium cybersecurity companies to minimize the impact of hacker exploits. With increasing health and financial implications of cyber attacks, technology needs to be enhanced to defend against malware. Using known vulnerabilities and countermeasure effectiveness as Quantum Save optimized the mitigation of a given exploit via the implementation of specific controls while remaining under a budget. By analyzing the distribution of the algorithm’s output at each budget range, there was strong evidence which suggested that Quantum Save increased mitigation from attacks and reduced monetary loss faced by a company. On top of that, Quantum Save scales linearly with $\mathcal{O}(n)$ complexity, making it useful in producing recommendation lists with 50-100 countermeasures. Quantum Save’s minimal run time and enhanced accuracy can help small-medium companies save millions of dollars by effectively allocating cybersecurity budgets.
\end{abstract}